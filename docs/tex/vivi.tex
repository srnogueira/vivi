\documentclass[10pt,twoside,a4paper]{report}
\usepackage[a4paper,margin=2cm]{geometry}
\usepackage{amsmath}

\begin{document}
\title{vivi: documentation}
\author{Rafael Nogueira Nakashima}
\date{August 2023}

%\frontmatter
\maketitle

%\mainmatter

\chapter{The steady-state problem}

Define the problem in steady-state conditions

\chapter{The multi-period problem}

Define the problem in multi-period formulation

\chapter{The optimization problem}

This chapter structured as:
\begin{enumerate}
\item Objective function
\item Heat cascade constraints
\item Other constraints
\end{enumerate}

\section{The objective function}
For economic analysis, the net present value is choosen as the
objective function to be maximized by the optimization. Since this
tool is not aimed to do a precise economic analysis of chemical
processes, but rather provide some quick estimates for optimization, a
number of simplifications for the economic analysis are proposed:

\begin{itemize}
\item No land cost.
\item No working capital.
\item No salvage.
\item No taxes and depreciation.
\item No startup time.
\item No degradation (Yearly production/efficiency does not alter along the years of project).
\item No additional costs of manufacturing besides resources inputs.
  % \item Full production after construction period (startup).
  % \item Yearly production/efficiency does not alter along the years of project.
\end{itemize}

Under these assumptions the net present value (NPV) of a chemical process
can be written as Eq. (\ref{eq:npv}):
\begin{equation}
  \label{eq:npv}
  % NPV = (R+L+O)\beta_{p/a}\beta_{p/f,s} + \sum_n^{n_t}C_{n}\beta_{p/f,n} % Old formulation, more complicated
  NPV = C_{r}\beta_{p/a} + \sum_{n=0}^{n_t}C_{n}\beta_{p/f,n} % New, simpler and inline with literature
\end{equation}

\begin{tabular}[h]{ll}
  NPV & Net present value, USD\\
  $C_{r}$ & Resource cost balance, USD/y\\
  %L & Direct labor costs, USD/y\\
  %O & Other operating expenses, USD/y\\
  $\beta_{p/a} $& Present value of an annuity, y\\
  %$\beta_{p/f,s} $& Single payment present value factor (startup year), -\\
  $\beta_{p/f,n} $& Single payment present value factor (year ``n''), -\\
  $C_{n}$ & Capital investment on year ``n'', USD/y \\
  $n_t$ & Total project years, y \\
  \\
\end{tabular}

\begin{equation}
  \beta_{p/a} = \frac{(1+i)^{n_t}-1}{i(1+i)^{n_t}}
\end{equation}

\begin{equation}
  \beta_{p/f,n} = \frac{1}{(1+i)^n}
\end{equation}


The resources produced or consumed by each technology (as well as the
ones defined in the boundary conditions) may vary by hour, as denoted
in Eq. (\ref{eq:var}). In most of the cases, average rates can be assumed for the
system, which can be multiplied by the operating hours in a year
(Eq. \ref{eq:var2}).

\begin{equation}
  \label{eq:var}
  C_{r} = \sum_t^{t_t}\sum_i^{i_t}(r^{out}_{i,t}c_{i,t}^{out}-r^{in}_{i,t}c_{i,t}^{in})
\end{equation}

\begin{equation}
  \label{eq:var2}
  C_{r} = \sum_i^{i_t}(r^{out}_ic_i^{out}-r^{in}_ic_i^{in})t_{op}
\end{equation}

\begin{tabular}[h]{ll}
  $r_{i,t}^{out}$,$r_{i,t}^{in}$ & Output or input rate of resource ``i'' at time ``t'', ex: kg/h.\\
  $c_{i,t}^{out}$,$c_{i,t}^{in}$ & Output or input specific cost of resource ``i'' at time ``t'', ex: USD/kg.\\
  $t_{op}$ & Operating hours in a year, h/y.\\
  \\
\end{tabular}


%Direct labor costs (L) are estimated based on the number of operators
%per shift ($N_{OL}$) and the yearly salary of an operator ($C_{OL}$),
%as proposed by Turton, et. al [REF]:

%\begin{equation}
%  \label{eq:labor}
%  L = 4.5 (N_{OL}C_{OL})
%\end{equation}

Capital investments for each year ($C_i$) are estimated from the fixed
capital investment of each technology ($C_{FCI}^\tau$) and additional
costs related to the heat exchanger network ($C_{HXs}^\tau$). In the
proposed Eq. (\ref{eq:capital}), the alpha factor refers to the
percentage of the FCI which is expended on a given year for the
replacement of certain components (e.g., catalysts). The fixed capital
investment of every unit should be estimated by the user, with the
exception of the heat exchanger network ($C_{HXs}^\tau$), which can be
estimated by vivi (Subsection X).

\begin{equation}
  \label{eq:capital}
  C_i =
  \begin{cases}
    \sum_\tau^{\tau_n} (C_{FCI}^\tau+C_{HXs}^\tau) &\text{if $n=0$}\\
    \sum_\tau^{\tau_n} (C_{FCI}^\tau\alpha_{i}^\tau) &\text{otherwise}
  \end{cases}
\end{equation}

\begin{tabular}[h]{ll}
  $C_{FCI}^\tau$ & Fixed capital investment (FCI) of technology ``$\tau$'', USD\\
  $C_{HXs}^\tau $ & Fixed capital investment of heat exchangers required for integration, USD \\
  $\alpha_i^\tau$ & Percentage of FCI of technology ``$\tau$'' reinvested in year ``i'', USD\\
  \\
\end{tabular}

%\begin{equation}
%  \label{eq:capital}
%  f(n) = \begin{cases}
%    n/2       & \text{if } n \text{ is even}\\
%    -(n+1)/2  & \text{if } n \text{ is odd}
%  \end{cases}
  
  % if 0 , then is FCI + HXs
  % if not, then it is a portion of the FCI.

%\end{equation}

% Here there is a problem, which cost correlation should be assumed for the HXs?
% Also, there may be instances where the heat exchangers are different or multiple equations may be necessary.
% Here the heat exchanger function is non-linear, therefore a subroutine to linearize would also be necessary.
% Possible problems in the definition of alpha including or excluding the HXs.

%Other manufacturing costs are assumed to be proportional to resources,
%labor and capital investments, as described in Eq. (\ref{eq:fix}):

%\begin{equation}
%  \label{eq:fix}
%  O = R\cdot f_{COM,R}+L\cdot f_{COM,L}+\sum_\tau^{\tau_n}C_{FCI}^\tau\cdot f_{COM,FCI}
%\end{equation}

For non-economic problems, discount rate, labor, capital investments
and other operating expenses can be set to zero. The objective
function then becomes the net balance of resources (R), which can be
set to a common energy/exergy basis by using the appropriated specific
costs.

% Default values for vivi

\subsection{Piece-wise linearization of capital investments}
% Give points and the software will assume linear between them

Specific equipment costs usually reduce with larger sizes, a concept
commonly refered as ``economy of scale'' [Turton]. As a consequence,
the capital investment of a process plant follows a non-linear
relationship with size. Thus, in order to include the capital
investments in a linear optimization problem, as proposed in vivi, it
is necessary to create a linear approximation of the ``economy of
scale'' for every technology.

% Why to force the problem to be a linear one? Integration with scheduling problems.

There are two main shortcommings in this approach: (1) the precision
of the cost approximation (2) the inclusion of binary variables in the
problem. Both problem can be minimized by carefully choosing the
smallest number of regions and/or limiting the search space. It is
important to highlight that the uncertainties in the CAPEX estimates
are usually much higher than the losses in precision by the
linearization.

% How would one write it in the optmization problem? Ex. Voll

\begin{equation}
  \label{eq:costLin}
  C_{FCI}^\tau = \sum_s^S(f_s^\tau a_s + y_s^\tau b_s)
\end{equation}

\begin{equation}
  \label{eq:costLin2}
  f_{s,min}y_s \leq f_s^\tau \leq f_{s,max}y_s
\end{equation}

\begin{equation}
  \label{eq:costLin3}
  f^\tau = \sum_s^S f_s^\tau
\end{equation}

\begin{equation}
  \label{eq:costLin4}
  \sum y_s^\tau \leq 1
\end{equation}
\begin{tabular}[h]{ll}
  $f_s^\tau$ & Size factor of technology ``$\tau$'' at segment ``s'' (continuous variable), -\\
  $a_s$ & Slope constant of linearization in segment ``s'', - \\
  $y_s$ & Binary variable associated with segment ``s'', - \\
  $b_s$ & Intercept constant of linearization in segment ``s'', - \\
  $f_{s,min}^\tau$ & Minimal size factor of technology ``$\tau$'' at segment ``s'' (continuous variable), -\\
  $f_{s,max}^\tau$ & Maximum size factor of technology ``$\tau$'' at segment ``s'' (continuous variable), -\\
  $f^\tau$ & Size factor of technology ``$\tau$'', -\\
  \\
\end{tabular}


\subsubsection{Getting linear parameters from data points}
This can be achieved by separating the search space into regions in
which the costs can be approximated as linearly proportional to the
size factors. For instance, given two data points (f,C) for size and
costs, a linear representation between them (segment ``s'') can be
written as:

\begin{equation}
  \label{eq:a}
  a_s=\frac{C_{i+1}-C_i}{f_{i+1}-f_i}
\end{equation}

\begin{equation}
  \label{eq:b}
  b_s=C_{i+1}-a_sf_{i+1}
\end{equation}

So for the segment ``s'' [$f_i$,$f_{i+1}$] the linear approximation
is given by:

\begin{equation}
  \label{eq:linear}
  C = a_sf+b_s
\end{equation}


\subsection{Estimation of heat exchanger costs}
% For every point given in the techs costs, calculate the heat exchanger costs and add to the value

There are a number of methods to estimate the cost of heat exchangers
and most of them rely on defining the pairs of hot and cold stream of
each equipment. In energy integration problems, the heat transfer
pairs are usually not defined in order to not restrict the
optimization problem. This imposes a challenge in economic
optimization, since CAPEX and OPEX are directly connected with the
heat exchanger costs and energy integration, respectively.

The approach proposed here is to estimate the heat exchanger costs by
costing every heat stream separately, based on the equivalent area
required to transfer heat at the global minimal temperature approach
(plus extra individual contributions):

\begin{equation}
  \label{eq:hx_cost}
  C_{HXs}^\tau = \sum_i C_{HX,i}^\tau = \sum_i F(A_{HX,i}) 
\end{equation}

\begin{equation}
  \label{eq:area}
  A_{HX,i} = \frac{1}{\Delta T_{min}+\Delta T_{extra}} \left ( \frac{Q_if^\tau}{h_i} \right )
\end{equation}

\begin{tabular}[h]{ll}
  $F()$ & Heat exchanger cost function, N/A\\
  $C_{HX,i}^\tau$ & FCI related to a heat stream ``i'' of technology ``$\tau$'', USD \\
  $A_{HX,i}^\tau$ & Heat exchanger area related to a heat stream ``i'' of technology ``$\tau$'', $m^2$ \\
  $\Delta T_{min}$ & Minimal temperature approach for heat transfer, K\\
  $\Delta T_{extra}$ & Extra temperature approach for heat transfer, K\\
  $Q$ & Heat transfer rate, W \\
  $h$ & Individual heat transfer film coefficient, $\frac{W}{m^2K}$ \\
  \\
\end{tabular}

This method is simple, systematic and captures the trade-offs between
CAPEX and OPEX of heat exchanger networks. However, it is also prone
to overestimate the costs, since heat transfer is rarely done at the
exact pinch temperature and the number of heat exchangers is doubled.

\subsubsection{Heat exchanger cost function}

To do (from book)

\section{Heat cascade constraints}

Each technology may have heat demands and/or requirements associated
with it. These heat transfers may be provided from other technologies
or by utilities, such as burners, cooling towers, among others. It is
important to size the technologies and utilities such as all energy
requirements are satified in the most efficient way.

Thus, pinch analysis is used to assess and size the technologies and
utilities in a process plants:

\begin{equation}
  \label{eq:pinch_1}
  R_{k-1,t}+\sum_\tau^{\tau_n}\sum_i Q_{i,k,t}^\tau = R_{k,t} \quad \forall k \in [1,...,K], \quad \forall t
\end{equation}

% This representation is actually for the case in which the non-discrete linearization.
% An alternative here is to express as Q_{i,k,t} and explain separately how one can solve this.
% Also a simplified version of the problem.

\begin{equation}
  \label{eq:pinch_2}
  R_{k,t}
  \begin{cases}
     = 0 & \text{if $k=0,K$}\\
    \geq 0 & \text{otherwise} \\
  \end{cases}
\end{equation}

\begin{tabular}[h]{ll}
  $R_k$ & Residual heat transfer of temperature interval ``k'', W\\
  $Q_{i,k,t}^\tau$ & Heat transfer of stream ``i'' at temperature interval ``k'' and time ``t'' from technology ``$\tau$'', W\\
  \\
\end{tabular}

It is important to note that this formulation of the heat cascade
constraint assumes that the temperatures (source, target) remain
constant under different loads of technology ``$\tau$''. In most cases
this is not the case, therefore this assumption should be taken
carefully.

Modular simplication
\begin{equation}
  Q_{i,k,t} = f^\tau_t Q_{i,k}^\tau
\end{equation}

Steady-state simplification
\begin{equation}
  Q_{i,k,t} = Q_{i,k}^\tau
\end{equation}


% \subsubsection{A possible alternative}

% \begin{itemize}
% \item Identify the fixed and variable heat streams.
% \item Fix the temperature intervals such as they represent the same
%   problem or a slighly worst version of it, but essentially that have
%   constant temperature intervals.
% \item Use piece-wise linearization to describe the variation of heat
%   at those intervals with load, similar to done with product
%   efficiency.
% \item Hope that the number of variables does not make the solution
%   infeasible to calculate. (this could be a parameter to change as
%   they do with mesh).
% \end{itemize}

\subsubsection{Determining temperature intervals}

There are different ways to determine the temperature intervals of a
pinch problem. In vivi, the rules proposed by Grimes [X] as described
by Papoulias and Grossmann [X] are slighly modified to include the
possibility of extra individual temperature contributions
($\Delta T_{extra}$). The proposed method is:

\begin{enumerate}
\item Subtract $\Delta T_{min}$ from all hot streams temperatures;
\item Subtract $\Delta T_{extra}$ from all hot streams temperatures;
\item Add $\Delta T_{extra} $ to all cold streams temperatures;
\item Place all supply temperatures in a decreasing order list;
\item Number the temperature intervals in increasing order
  (k=1, 2, ..., K) from each sequence of temperature pairs.
\end{enumerate}

% It is important to notice that this method is different from Klemes,
% since it only uses the source temperatures.

% Not completely sure it that is right.

%The method is analogous to the temperature contribution approach
%described by Klemes [X]. However, the temperature contribution method
%turns more difficult to estimate the heat transfer area, since the
%minimal temperature approach is dependent on two heat streams in that
%method.

\subsubsection{Determining  $Q_{i,k}^\tau$}
The heat transfer associated with stream ``i'' at the temperature
interval ``k'' can be determined by comparing the shifted temperatures
with the temperature interval pair. Here we refer to the lowest
temperature in the interval as $T_k^{low}$, while the highest
temperature is refered as $T_k^{high}$.

If both shifted temperatures are below $T_k^{min}$ or above
$T_k^{high}$, then $Q_{i,k}$ is null. Otherwise, the heat transfered
in the temperature interval can be calculated by the heat capacity of
the stream ($Cp_i$) and its temperature variation inside the
temperature interval.

\begin{equation}
  \label{eq:heat}
  Q_{k,i}=Cp_i(T_{i,k}^{high}-T_{i,k}^{low})
\end{equation}

\begin{equation}
  \label{eq:t_high}
  T_{i,k}^{high} =
  \begin{cases}
    \min(T_{s,i},T_k^{high}) & \text{if "i" is a hot stream}\\
    \min(T_{t,i},T_k^{high}) & \text{otherwise}\\
  \end{cases}
\end{equation}

\begin{equation}
  \label{eq:t_low}
  T_{i,k}^{low} =
  \begin{cases}
    \max(T_{t,i},T_k^{low}) & \text{if "i" is a hot stream}\\
    \min(T_{s,i},T_k^{low}) & \text{otherwise}\\
  \end{cases}
\end{equation}

% Figure from thesis can help

\section{Resources balance constraints}

Resources balance 
\begin{equation}
  \label{eq:resources}
  r_{i,t}^{in}+S_{t-1}+\sum_\kappa^{\kappa_n} (r_{i,t}^{\kappa,in}-r_{i,t}^{\kappa,out}) = r_{i,t}^{out}+S_t
\end{equation}

% Important: t refers to the end of the hour. Ex: end of first our.

A simplified version is:
\begin{equation}
  \label{eq:resources}
  r_{i}^{in}+\sum_\kappa^{\kappa_n} f^\kappa(r_{i}^{\kappa,in}-r_{i}^{\kappa,out}) = r_{i}^{out}
\end{equation}




Explain the numerous variables

Explain the storage variables
\begin{equation}
  S_{t-1} = \sum_\sigma^{\sigma_n}s_{i,t-1}^\sigma\eta_i^\sigma
\end{equation}
%(eta only models the self discharge)


\begin{equation}
  S_{t} = \sum_\sigma^{\sigma_n}s_{i,t}^\sigma
\end{equation}
\begin{equation}
  s_{i,t}^\sigma \leq f^\sigma
\end{equation}
% f_sigma assumes that the storage capacity does not change with time, which may happen


Explain the approximations possible for r's (simple or piece-wise)

\subsection{Modular approximation}
\begin{equation}
  r_t = f_t^\tau r_i
\end{equation}

\begin{equation}
  l_{min}f^\tau \leq f_t^\tau \leq l_{max}f^\tau 
\end{equation}

\subsection{Reformulation strategy}
(The same strategy would be necessary for heat)


For only one segment (Voll):
\begin{equation}
  r_t = \xi_t b_s + f_t a_s 
\end{equation}

\begin{equation}
  \xi_t l_{min} \leq f_t \leq \xi_t l_{max}
\end{equation}

\begin{equation}
  \delta_t f_{min} \leq \xi_t \leq \delta_t f_{max} 
\end{equation}

\begin{equation}
  (1-\delta_t)f_{min} \leq \Psi_t - \xi_t \leq (1-\delta_t)f_{max}
\end{equation}

\begin{equation}
  \Psi_t = f \quad \forall t
\end{equation}
\begin{equation}
  \xi_t \equiv  \delta_t f 
\end{equation}


For multiple segments (Li):

\begin{equation}
  r_t = \sum_s \left ( \xi_{t,s} b_s + f_{t,s} a_s \right ) 
\end{equation}

\begin{equation}
  \xi_{t,s} l_{s,min} \leq f_{t,s} \leq \xi_{t,s} l_{s,max}
\end{equation}

\begin{equation}
  \delta_{t,s} f_{min} \leq \xi_{t,s} \leq \delta_{t,s} f_{max} 
\end{equation}

\begin{equation}
  (1-\delta_{t,s})f_{min} \leq \Psi_{t} - \xi_{t,s} \leq (1-\delta_{t,s})f_{max}
\end{equation}

\begin{equation}
  \Psi_{t} = f \quad \forall t
\end{equation}

\begin{equation}
  \sum_s \delta_{t,s} \leq 1 \quad \forall t
\end{equation}

\begin{equation}
  \xi_{t,s} \equiv  \delta_{t,s} f 
\end{equation}

% Formulation given by Voll can be used here to derive the partial load operation of a certain resource.

\section{Other constraints}

\subsection{Boundary conditions}
Boundary conditions are the resource availability or demand for a certain time or period:

\begin{equation}
  r_{i,t} = r_{cte}
\end{equation}

\begin{equation}
  \sum_t r_{i,t} = r_{cte}
\end{equation}

\subsection{Ramping}
A common constraint in problems is the ramping limit of a certain
technology:

\begin{equation}
  \Delta_{down} f \leq f_t-f_{t-1} \leq \Delta_{up} f
\end{equation}

% Technology sizes are already included in the formulation of
% piece-wise linearization of costs or resource balances.

% What about startup and shutdown? Or other operating mode?
% - Is not much the goal to have a precise model, but to analyze the integration possibilities.

\subsection{No turnoff}
Changes constraint about the sum of binaries.

\section{Summary}

Show the whole optimization problem with all variables in the general sense for vivi


% DELETED

%\begin{equation}
%  \label{eq:fci}
%  C_{FCI}^\tau = (\alpha_{cont.}+\alpha_{fee})\sum_j C_{BM,j} + \alpha_{aux}\sum_j C_{BM,j}^0
%\end{equation}
% And this part here is the one which is linearized (which also have some problems, because it loses information)
%\appendix
%\backmatter


\end{document}
